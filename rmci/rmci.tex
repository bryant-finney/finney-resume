% file:     rmci.tex
% created:  2018-12-21 09:23
% author:   Bryant Finney <bryant.finney@uah.edu>
\documentclass[10pt,a4paper]{article}

% Comment the following line to deny the usage of umlauts and other non-ASCII characters
\usepackage[latin1]{inputenc}
\usepackage{amsmath}
\usepackage{amsfonts}
\usepackage{amssymb}
\usepackage{graphicx}
\usepackage{geometry}

\geometry{
  a4paper,          % redundant (already in \documentclass)
  left=0.5in,
  right=0.5in,
  top=0.5in,
  bottom=0.90in,
  heightrounded,    % better use it
}

% Uncomment the following line to allow the usage of graphics (.jpg, .png, etc.)
%\usepackage{graphicx}

% Start the document
\begin{document}

% Create a new 1st level heading
\section{Embedded Systems Engineer}
\begin{itemize}
  \item Developed and integrated shared libraries in C for communication and control of
    third-party tracker system
    \begin{itemize}
      \item Libraries utilized two communications channels over RS-485 and RS-232 for
        configuration transmission and data retrieval

      \item Protocols included ASCII and non-ASCII binary packets with checksums and
        ASCII hex representations

      \item Configuration parameters were retrieved from SQLite Database and ASCII
        configuration file

      \item Developed and integrated libraries for tokenizing and parsing ASCII file
        contents with comments
    \end{itemize}

  \item Integrated tracker communications, acquisition, and processing libraries with
    framework for collecting RTB data
    \begin{itemize}
      \item Developed lightweight logging library using variadic functions for creating
        and writing to log files, as well as writing messages to std-out for real-time
        feedback during developmental testing of framework libraries

      \item Assisted with aircraft installation and testing on the airframes OH-58C and
        OH-6A of the entire embedded system in aspects of vibration data collections and
        rotor track height data collections
    \end{itemize}

  \item Developed and tested vibration data analysis algorithms for condition-based
    maintenance from an embedded OS
    \begin{itemize}
      \item Implemented the algorithms in C for utilization by an ARM cortex 335x series
        microcontroller. The focus of the algorithms was on time domain, frequency
        domain, and synchronous domain analyses

      \item Developed a high-level framework in C incorporated with the Yocto project
        for handling onboard processing
        \begin{itemize}
          \item Designed and integrated communication functions with an embedded SQLite
            database for retrieving algorithm parameters, as well as loading and storing
            both raw and processed data

          \item Interfaced with low-level routines for collecting raw data and
            monitoring system state via a watchdog timer

          \item Established communications protocol for system-wide integration in order
            to perform all onboard data acquisitions and processing upon triggers either
            by the user or by flight regime
        \end{itemize}

      \item Performed software verification testing by comparing results from the data
        analysis framework with a Python transliteration of the original MatLab scripts.
        The results were displayed and analyzed
    \end{itemize}

  \item Designed and tested firmware for the PIC24F08KL301 MCU using MPLAB X, Eclipse,
    and PICKIT3
    \begin{itemize}
      \item The microcontroller was used to emulate tachometer signals with two 8-bit
        timers and to translate between RS-485 and RS-232 message protocols for
        interfacing with a rotor blade tracking system

      \item The two UART peripherals were utilized for the serial communications: one
        dedicated to RS-485 communication with the controlling OS, and the other was
        dedicated to RS-232 communication with the tracking system

      \item Implemented a communications protocol for responding to various control
        bytes transmitted over RS-485

      \item Developed a boot loader for executing firmware updates over serial traffic:
        created host side in Python using minimal packages for ease of deployment on the
        embedded Linux system, and tested the MCU side written in C

      \item Phase 1 development was done on the SAMD21G18A MCU with Eclipse and the
        Arduino IDE for programming
    \end{itemize}

  \item Led a multidisciplinary team to design a new product for tracking helicopter
    rotor blades

  \begin{itemize}
    \item Performed individual research and bench-marking, theoretical modeling using
      linear algebra and geometry, and data visualization for demonstrating theoretical
      system performance characteristics. Provided the recommendation for the selected
      design and continued to lead the development of this design

    \item Conducted electro-optical radiometric modeling of solid state sensors for
      blade detection

    \item Completed optical design of system hardware using matrix ray tracing in
      MathCAD, Python, and Matlab

    \item Developed opto-mechanical design of illuminator and sensor to
      minimize space requirements while considering reliability, affordability, and
      maintainability

    \item Reviewed materials and manufacturing processes for component design and
      selection

    \begin{itemize}
      \item Design criteria included the following: overall size, weight, and cost; IR
        safety compliance according to IEC 60825-1 and IEC 62471; insensitivity to
        environmental lighting conditions; environmental qualification according to
        DO-160 and MIL-STD-810; hardware reliability, aesthetic appeal, and convenience
        of operation to the end user
    \end{itemize}

    \item The system is currently entering the prototype stage and is estimated to cost less than
      existing systems while improving performance and reducing maintenance costs
  \end{itemize}

  \item Performed all engineering analysis, design, and fabrication of a stand for testing
    developmental systems

  \begin{itemize}
    \item The stand replicated a scaled rotor of approximately 8 ft. diameter, capable of
      rotational speed over 300 rpm

    \item Used finite airfoil drag analysis to model the transient and steady-state speed based
      on input power. Total cost was under \$1000, and the test stand was used for rotor track
      and balance tests
  \end{itemize}
\end{itemize}
% End of the document
\end{document}
